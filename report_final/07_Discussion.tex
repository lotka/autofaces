\chapter{Discussion \& Conclusion}

  \section{Evaluation of the Method}
    These models present a very large hyper parameter space that can be explored, therefore
    it is no surprise that this project was able to explore a very small part of this, using
    existing work and experiments for guidance.

    The path taken in this project was one where a classifier network was built
    with some inspiration from the literature and then the decoder section was added on.
    The inverse approach may have given better results, which is to design a fully functional
    autoencoder and then add a classifier.

    Further it may have been instructive to fully explore what features the autoencoder is learning
    by visualising the convolutional filters and outputs. This is fairly time consuming and
    so was not included, it could be that the features are not directly correlated to AU's and
    are much more related to reconstructing subject related features.

    Certain things could have been improved with the preprocessing methods,
    potentially doing them before resizing the images may have improved their
    ability to expose AU features. Lots of time was also spent on these, which could have
    possibly been spent on improving the balance of that data.

    A key issue with the DISFA dataset is that many frames do not contain much information
    because of neutral faces. Increasing the frequency of the high information labels
    might have lead to better generalisation for the test data set.

    The training of the autoenocoder was not stacked, instead it was joint which has
    proven to be preferable in certain situations \cite{Zhou2014}. However, in this case
    it might have been more reliable to perform stacked training of the autoencoder to ensure
    useful features were picked out.

  \section{Future Work}
    This project effectively built the neural network structures from scratch.
    An alternative approach would be to find a network in the literature which is pre-trained
    and already is very good at extracting certain features and use that as a base.
    This also might have allowed for the easier comparison of the results to the literature.

    Artificially increasing the size of the training set seems like a big priority,
    this would allow both the classifier and autoencoder to learn more general features.
    Some methods for doing this are as follows:

    \begin{itemize}
      \item Courrupting the input image or hidden layer representations
      \item Appyling random transformations to the input image (crops, displacements, etc.)
      \item Training the autoencoder with other datasets
    \end{itemize}

    % Using a ResNet \cite{} might have allowed larger networks with the penalty of an increased
    % nubmber of parameters.

  \section{Conclusion}
    The results have not shown that an autoencoder, in our particular setting,
    gives significant improvements to classification performance. However it has
    explored how preprocessing techniques and various neural network structures
    interact, showing that with small datasets such as DISFA other techniques may
    be more important. An unconventional part of this work, given the excitement
    in the field of deep learning is that the smaller networks perform better.
    This is most probably due to the fact that the input data was too homogeneous
    and that the task of detecting AUs is difficult, in particular in the way the problem was set up
    with even intensity one AUs (barely visible to humans and related to context)
    were included as a positive example. The method of per subject mean face normalisation was
    found to out perform other preprocessing methods conclsuively and the classifier
    achieved competative results on the DISFA dataset.
