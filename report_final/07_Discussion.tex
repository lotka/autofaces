  % The Four Pillars of Autofaces
  %%%%%%%%%%%%% The variation between images %%%%%%%%%%%%%%%%%%%%%%%%
  %%%%%%%%%%%%% They contain fewer images %%%%%%%%%%%%%%%%%%%%%%%%%%%
  %%%%%%%%%%%%% The possibility for AUs to occur simultaneously %%%%%
  %%%%%%%%%%%%% Non-uniform AU occurrence %%%%%%%%%%%%%%%%%%%%%%%%%%%
\chapter{Discussion}


  \section{Methodology}
    % - organisation of code and data was good, maybe a bit too late in the day
    %   - good to automate comparison
    % - experimentation structure
    % - iterations posed an issue throughout
    % - suffered from overhead of reproducing existing technique, ideal situation would be to start from something state of the art
    % - suffered from non-deterministic computations, should have used a more mature framework. This might have allowed
    %   above issue to be solved
    % - Limited time so couldn't test more structures
    % - Talk about the fact that the direct inversion of functioning classifiers was the wrong approach,
    %   it should have been to apply state of the art autoencoders and turn them into classifiers
    % - Maybe these classifiers are too big for the purpose.

    This section discusses some of the problems encountered and solutions found
    to issues relating to methodology.

    The structures in this report allow for the exploration of a hyperparameter
    space which is very large. Each parameter has complex non-linear interactions with
    others, hence it is difficult to explore all options and configurations.
    This work has explored a small section of this space with the guide of experimentation
    and existing literature.

    Good organisation of code and data allowed for the easy running
    and comparison of experiments which could take up to a day. These were easily transferred
    into the graphs in this report. Automating many of these tasks therefore seems worthwhile,
    as once the initial overhead of implementation and testing is passed, they encourage experimentation.

    \subsection{Validity of results}

    \begin{itemize}
      \item \textit{Training iterations: }
        The number of iterations used to train a model posed a challenge. Different networks
        and different configurations all need a different number of iterations to produce their
        best possible model. The approach of this work was to fix the number of iterations
        and use the fact that larger models need more iterations to their disadvantage.
        It seems this did not cause a problem, because overfitting was not massively detrimental to the final score, each network could recieve enough iterations to converge.
      \item \textit{Overhead in reproducing state of the art:}
        There was an overhead in trying to reproduce existing work in the literature
        and only then to implement something new. It would have potentially been easier
        to start from an existing state of the art model, however none were openly available
        for the exact required application, many existed for other image classification tasks.
        This was avoided because inverting very deep convolutional networks might be difficult
        and not achieve good performance as training such structures is difficult.
        The approach of taking classification networks from the literature and then
        inverting them to create autoencoders was potentially not ideal, as they
        were not designed with this purpose in mind. Advanced techniques to create robust
        autoencoders might have been a better starting point, this is further discussed in future work.
      \item \textit{Non-deterministic computations:}
        The ability to compare networks was reduced by the fact that TensorFlow could
        not perform deterministic computations even when random number generators were
        seeded.
      \item \textit{Visualisation of convolutional layers:}
        This provided a good way to diagnose and understand what the final to networks
        were learning and helped reach more concrete conclusions.
    \end{itemize}
    % This would have posed some issues however, as inverting a very deep network
    % to create an autoencoder might not have been feasible.



  \section{Results}
    One of the goals of the project was to find a way to increase the variation
    between images in the DISFA dataset in order to allow improved training of the network.
    As is shown in the results (section \ref{tab:psearch}), this is the variable that has
    the greatest effect on classification and autoencoding performance.

    Joint classification of AUs was tackled by proposing the Binary Softmax layer,
    which proved to perform better than the other available solutions.

    The results show that there are cases where both the autoencoder
    and classifier can retain their functionality,
    this is shown in the L2 Regularisation section
    and when both objectives were trained
    equally with constant balance. However there is little evidence that the autoencoder structure which was used
    helps classification. This is most likely because the autoencoder failed to learn enough features.



  \section{Future Work}





    \begin{itemize}
      \item \textit{Larger fully connected models} - Instead of using convolutional networks, the DISFA dataset might have been
                                                     modelled well with just fully connected layers and regularisation techinques.
                                                     This was ignored as the literature mainly used convolutional layers, but might hold
                                                     interesting insights, in particular when combined with the proposed autoencoder classifier structure.
      \item \textit{Start from autoencoders:} -
            Instead of starting from classification structures, state of the art autoencoders
            should be the starting point. These could include the follow techniques stacked architectures\cite{Zhou2014}, denoising architectures\cite{stacks,Vincent2008a} or variational autoencoders\cite{Kingma2013}.
      \item \textit{Improved visualisation of autoencoder features:} As in \cite{Khorrami2015} the maximal excitation for different neurons in convolutional layers
              could be found in order to gain greater insight into what features are being extracted.
      \item \textit{  Artificially increasing the size of the training set:}
            this would allow both the classifier and autoencoder to learn more
            general features. Some methods for doing this are as follows:

            \begin{itemize}
              \item Appyling random transformations to the input image (crops, displacements, etc.)
              \item Training the autoencoder with other datasets
              \item Including high intensity AU examples more often
            \end{itemize}
      \item \textit{ Perform a cross validatation:} this would allow the comparison of the results to the literature.
      \item \textit{ Incorporating temporal information:} facial expressions depend on context
            and are not of constant intensity in time. This information could be incorporated into
            a recurrent neural network structure as in \cite{Jaiswal2016}
    \end{itemize}

\chapter{Conclusion}
  The results have not shown that an autoencoder, in our particular setting,
  gives significant improvements to classification performance. However it has
  explored how preprocessing techniques and various neural network structures
  interact, showing that with small datasets such as DISFA other considerations
  are also important. An unconventional part of this work, given the
  excitement in the field of deep learning is that the smaller networks
  perform better. This is most probably due to the fact that the input data
  was too homogeneous and that the task of detecting AUs is difficult, in
  particular in the way the problem was set up where even intensity one AUs
  (barely visible to humans and related to context) were included as a
  positive example. The method of per subject mean face normalisation was
  found to out perform other preprocessing methods conclusively and the
  classifier achieved competitive results on the DISFA dataset. Lastly the Binary Softmax
  classifier proved to be a useful structure for jointly classifying AUs, giving a maximum average classification ROC of 0.83.
